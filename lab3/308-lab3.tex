\documentclass[letterpaper,10pt]{article}
\usepackage{amsfonts}
\usepackage{amssymb}
\usepackage{amsmath}
\usepackage{amsthm}
\usepackage{enumitem}
\usepackage{mathrsfs}
\usepackage[utf8]{inputenc}

\addtolength{\voffset}{-1cm}
\addtolength{\hoffset}{-2cm}
\addtolength{\textwidth}{4cm}
\addtolength{\textheight}{2cm}
\newcommand{\cmd}[1]{\texttt{#1}}

\usepackage[pdftex]{color}
\usepackage[pdftex]{graphicx}
\usepackage[pdftex]{hyperref}
	\hypersetup{colorlinks}
	\hypersetup{linkcolor=blue}

\title{CprE 308 Laboratory 3: Concurrent Programming with the pthreads Library}
\author{Department of Electrical and Computer Engineering \\ Iowa State University}
\date{Spring 2014}

\begin{document}
\maketitle

\section{Submission}

Include the following in your submission:

\begin{itemize}
 \item A summary of what you learned in the lab session. This should be no more than two
 paragraphs. Try to get at the main idea of the exercises, and include any particular
 details you found interesting or any problems you encountered.

 \item A write-up of each experiment in the lab. Each experiment has a list of items
 you need to include. For output, cut and paste results from your terminal and
 summarize when necessary. Include all relevant details.

 \item Submit your summary and write-ups in class.  Be ready to demo your programs to your TAs at the beginning of the next section.

\end{itemize}

\section{Instructions}
\begin{enumerate}
 \item Read the \emph{entire} lab description first! Section~\ref{hints} in particular will help you finish the other exercises.
 \item Manual pages of all different functions that you encounter.
 \item Read the following man page: \verb!man pthread_cond_wait!
 \item Refer to Section \ref{hints} in this lab description for further hints on programming.
\end{enumerate}

\section{Excercises}
\subsection{Programming with pthreads}
In this exercise you will learn about concurrent programming using
pthreads. We will step through a simple program using pthreads:

\begin{itemize}
 \item Open a new file and include the \cmd{pthread.h} file.
 Create two new function prototypes called thread1 and thread2 (i.e.
 ``\verb+void* thread1();+''). These will be the methods that our
 pthreads will run. Look at the example code
 in Section \ref{hints}, or in t1.c, and t2.c.

 \item Code the main function. Make two instances of the data
 type \verb=pthread_t= called ``i1'' and ``i2''.

 \item Create the threads using the \verb=pthread_create= function. Create both threads,
 passing similar arguments as shown in the example in the help section. The
 last argument may be passed a value of NULL, since we don’t pass any
 arguments to our thread1 and thread2 functions.

 \item After the main function write the thread1 and thread2 functions.
 Just have each function print a message to the screen identifying itself
 (i.e. \verb+printf("Hello,I am thread 1\n");+). Add a print statement in the end
 of the main function (i.e. \verb+printf("Hello,I am main process\n");+).

 \item Compile the program using the \cmd{-lpthread} option.
\begin{verbatim}
$ gcc -o ex1 -lpthread ex1.c
\end{verbatim}
This option compiles the code linked with the pthread library.

\end{itemize}

Run the program and observe the output. What happens? Well, we created the threads, but we forgot to
use the \verb+pthread_join+ function to allow them to finish before main terminates.

\begin{enumerate}
\item (6 pts ) To make sure the main terminates before the threads finish,
 add a sleep(5) statement in the beginning of the thread functions.
 Can you see the threads' output? Why?
\item (2 pts ) Add the two \verb=pthread_join= statements just before the printf statement in
 main. Pass a value of NULL for the second argument.
 Recompile and rerun the program. What is the output? Why?
\item (2 pts ) Include your commented code with your submission.
Label it ``pthread example''.
\end{enumerate}


\subsection{Thread synchronization}
In this experiment you will learn how to synchronize threads that share a common
data source. Mutexes and conditional variables are used
as synchronization mechanisms for pthreads.

\subsubsection{Mutex}
We will first look at code examples that illustrate how threads
use mutexes to exclusively access critical area. Download t1.c from
the class website and examine its contents.

\begin{enumerate}
\item (2 pts ) Compile and run t1.c, what is the output value of v?
\item (8 pts) Delete the \verb+pthread_mutex_lock+ and \verb+pthread_mutex_unlock+ statement
in both increment and decrement threads. Recompile and rerun t1.c,
what is the output value of v?
Explain why the output is the same, or different.
\end{enumerate}


\subsubsection{Conditional Variable}
Next download t2.c. This is a ``Hello World" program using threads.
The program uses threads to print ``hello world". The thread that prints
``world" waits for the other thread to finish printing ``hello". This
is achieved using condition variables.

\begin{itemize}
\item Modify the program by adding another thread (and routine)
called ``again" Use a second conditional variable to synchronize
the three threads so that they print out the statement ``Hello World Again!"

\item To implement this correctly, you must understand why the ``done" flag
is necessary. Think about the case where the hello function runs first and
sends the signal before world is waiting (you can use a sleep statement to
force this case). Note that a signal is not received unless someone is waiting
for it first. Could the world thread sleep forever? When you make your
changes, take this problem into account.
\end{itemize}

\begin{enumerate}
\item ( 10 pts ) Include your modified code with your lab submission and comment
on what you added or changed. Label this ``t2.c".
\end{enumerate}


\subsection{Modified Producer Consumer Problem}

Next, you will work on a producer-consumer type problem that is different
from what we discussed in class. So, please read the description carefully.

Download t3.c from the class website. The goal of this program is to run a
group of consumers and a single producer in synchronization. The program will
start one producer thread, which runs the function ``producer", and many
consumer threads, each of which runs the function ``consumer".

The producer should produce items only when the number of items in the supply has
reached zero. Until this happens, the producer waits. The producer produces 10
items each time. When there are no more consumer threads remaining to consume
items, the producer must exit.

Each consumer thread waits until there is at least one item of supply remaining
to consume. It then consumes one item of supply, and then exits.

Your task is to fill in the code for the producer. The code for the consumer
has already been filled in.

\begin{enumerate}
\item ( 20 pts ) Include your modified code with your lab submission and
comment on what you added or changed. Label this ``t3.c".
\end{enumerate}


\section{Hints}
\label{hints}

\subsection{Creating pthreads}
The function used to create a pthread is:
\begin{verbatim}
int pthread_create(pthread_t * thread, pthread_attr_t * attr,
    void * (*start_routine)(void *), void * arg);
\end{verbatim}
This will start a thread using the function pointed to by \verb+start_routine+, which is a function
pointer. You can think of a function pointer as the starting address of a function. The
name of a function in C will act as a function pointer. Note the \verb+pthread_create+'s
prototype calls for a function pointer to a function that takes a void pointer and returns a
void pointer. If the type is void, how are we going to pass data into the thread?
The argument of ``\verb+start_routine+'' is separately passed through ``arg''. If \verb+start_routine+ needs
more than one argument, ``arg'' should point to a structure that encapsulates all of
the arguments. The code below shows how to typecast the structure to and from a void
pointer so the \verb+pthread_create+ function can be used correctly.
\label{createex}
\begin{verbatim}
#include <pthread.h>

struct two_args {
    int arg1;
    long arg2;
};

void *foo(void * p) {
    int local_arg1;
    long local_arg2;
    struct two_args * local_args = p;
    local_arg1 = local_args->arg1;
    local_arg2 = local_args->arg2;
    /* continue work */
}

int main() {
    pthread_t t; /* t: identifier for thread */
    struct two_args* ap; /* a pointer to the arguments for "foo" */
    ap = malloc(sizeof(struct two_args));
          /* sizeof returns the number of bytes in the structure */
    ap->arg1 = 1;
    ap->arg2 = 2;

    pthread_create(&t, NULL, foo, (void*)ap);
    /* continue work */
}
\end{verbatim}



\subsection{Conditional Variables}

\begin{itemize}
\item Condition variables allow us to synchronize threads by having them wait
until a specific condition occurs. In t2.c, the ``world" thread waits until the ``hello"
thread activates the condition. Once the \verb+pthread_cond_wait+ statement is active,
it automatically releases the mutex and waits until it receives an active signal.
When the condition receives its signal, the \verb+pthread_cond_wait+ function is able to
lock the newly available mutex. Assuming that the mutex can be locked the
thread continues executing. If the mutex cannot be locked the thread waits
for ``an unlock" to occur.

\item A helpful construct for waiting might look like this:
\begin{verbatim}
pthread_mutex_lock(&mutex)
while (can’t proceed)
    pthread_cond_wait(&cond, &mutex)
pthread_mutex_unlock(&mutex)
\end{verbatim}
You must have a mutex locked before waiting. Also, note that while the thread
is waiting, the mutex is unlocked. It will be locked again when
the thread resumes.

\item You will find that there are two ways to signal threads waiting on
a condition variable: \verb+pthread_cond_signal+, and \verb+pthread_cond_broadcast+. The latter
will allow all threads currently waiting to continue, one after the other. A
condition variable signal will be received only if another thread is already waiting.

\end{itemize}


\subsection{Compilation}

You need to pass the thread library on the \cmd{gcc} or \cmd{g++}
command line with the option \cmd{-lpthread}.
\begin{verbatim}
$ gcc -lpthread -o threads threads.c
\end{verbatim}


\end{document}
