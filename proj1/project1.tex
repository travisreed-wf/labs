\documentclass[letterpaper,10pt]{article}
\usepackage{amsfonts}
\usepackage{amssymb}
\usepackage{amsmath}
\usepackage{amsthm}
\usepackage{enumitem}
\usepackage{mathrsfs}
\usepackage[utf8]{inputenc}

\addtolength{\voffset}{-1cm}
\addtolength{\hoffset}{-2cm}
\addtolength{\textwidth}{4cm}
\addtolength{\textheight}{2cm}
\newcommand{\cmd}[1]{\texttt{#1}}

\usepackage[pdftex]{color}
\usepackage[pdftex]{graphicx}
\usepackage[pdftex]{hyperref}
	\hypersetup{colorlinks}
	\hypersetup{linkcolor=blue}

\title{CprE 308 Project 1: UNIX Shell}
\author{Department of Electrical and Computer Engineering \\ Iowa State University}
\date{Spring 2014}

\begin{document}
\maketitle

This is a two-week programming project assignment, starting in the week of Feb. 10-14, 2014, 
and will due in the week of Feb. 24-28, 2014.

\section{Submission}
Submit to Blackboard system by due date (starting time for your lab session, Feb. 25, 26, and 28, respectively, 2014). 
Include the following:
\begin{itemize}
 \item A cohesive summary of what you learned in the project. This should be no more than two paragraphs.
 \item A complete listing of your source code; commented and formatted neatly. Use good programming practices
  (create methods to abstract large bodies of code and code common to multiple methods, only use global
  variables when you must, and check syscall or system dependant libcall return values for errors – don't assume
  the OS will always succeed in fulfilling your requests.)
 \item A Makefile to compile your code. The grader should be able to compile your project by issuing the make command. If needed, please include
 the instructions to compile the source (in a comment at the top of the .c file), or in the Makefile.
 \item The assignment should be submitted through Blackboard before the deadline. Your submission should 
 be a single tar archive containing your source code, your report, and your Makefile. 
 Your archive should be named shell combined with your name and lab section (e.g. shell-lastname-A.tar.gz). 
 To create an archive, simply run the command ``{\em tar -czvf shell-lastname-A.tar.gz folder/ }", 
 where {\em folder/ } is the folder you wish to archive.
\end{itemize}

\paragraph{Late Policy}
You will lose 10\% per day for one week. After one week you can no longer receive any credit.

\paragraph{Grading Criteria:}
\begin{itemize}
 \item Summary –- 15\% (Well written and specific).
\end{itemize}

Submitted Source Code
\begin{itemize}
 \item Compiles –- 10\% 
 \item Can be read and understood easily -- 10\%
 \item Performs the required functionality –- 65\%
\end{itemize}

\section{Project}
In this project you will be creating your own version of a UNIX shell. It will not be as sophisticated as \emph{bash},
the shell you have been using in lab, but it will perform similar functions.
A UNIX shell is simply an interactive interface between the OS and the user.
It repeatedly accepts input from the user and initiates processes on the OS based on that input.

\subsection{Requirements}
\begin{enumerate}
 \item Your shell should accept a \verb!-p <prompt>! option on the command line when it is initiated. The \verb-<prompt>- option
       should become the user prompt. If this option is not specified, the default prompt of ``\verb'308sh> ''' should be used.
       Read section \ref{cmdline} on command line options for more information.
 \item Your shell must run an infinite loop accepting input from the user and running commands until the user
       requests to exit.
 \item Each line of user input should be treated as either a builtin command, or a command to be executed. See section
  \ref{cmds} for a list of the builtin commands your shell should support.
 \item For each executed command, the shell should spawn a child process to
       run the command. The shell should try to run the command exactly as typed. This will require the user to know
       the full path (absolute \cmd{/some/dir/some/exec} or relative \cmd{../../some/exec}) to the executable, unless the
       executable they wish to run is in one of the directories listed in the PATH environment variable.
       HINT: execvp() will search PATH for you.
 \item The shell should notify the user if the requested command is not found and cannot be run.
 \item For each spawned child, print the process ID (PID) before executing the specified command.
       It should only be printed once and it must be printed before any output from the command.
       You can include other information (name of program, command number, etc) if you find it useful.
 \item By default, the shell should block (wait for the child to exit) for each command. Thus the prompt will not be
       available for additional user input until the command has completed. Your shell should only wake up when the
       \emph{most recently executed} child process completes. See waitpid.
 \item The shell should evaluate the exit status of the child process and print the conditions under which it exited
       (identify which process exited by its PID). Read the man page for wait to see a list of macros that provide this
       information (man 2 wait). Two examples of exit status are normal exit and killed (signaled).
 \item If the last character in the user input is an ampersand (\&), the child should be run in the background (the
       shell will not wait for the child to exit before prompting the user for further input). Remove the \& when
       passing the parameters to exec. When your background process does exit, you must print its status like you do
       for foreground processes. 
       
       An example is, if the user entered ``/usr/bin/emacs” then the shell would wait for 
       the process to complete before returning to the prompt. However, if the user
       entered ``/usr/bin/emacs \&” then the prompt would return as soon as the process was initiated.

       HINT: to evaluate and print the exit status of a background child
       process call waitpid with -1 for the pid and WNOHANG set in the options. This
       checks all children and doesn’t block – see the man page. To do this periodically,
       a check can be done every time the user enters a command.
\end{enumerate}

\subsection{Builtin Commands} \label{cmds}
The following commands are special commands (also called built-in commands) that are not to be treated as programs to be launched.
No child process should be spawned when these are entered.

\begin{itemize}
 \item \cmd{exit} -- the shell should terminate and accept no further input from the user
 \item \cmd{pid} -- the shell should print its process ID
 \item \cmd{ppid} -- the shell should print the process ID of its parent
 \item \cmd{cd <dir>} -- change the working directory. With no arguments, change to the user's home directory
  (which is stored in the environment variable HOME)
 \item \cmd{pwd} -- print the current working directory
 \item \cmd{set <var> <value>} -- sets an environment variable (which is visible in all future child processes).
  If there is only one argument, clears the variable.
 \item \cmd{get <var>} -- prints the current value of an environment variable
\end{itemize}

\subsection{Extra Credit}
\begin{itemize}
 \item  Keep track of the names of your child processes and output them with their PIDs (i.e. ``process pid
  (procnam) ''). For this, you need a data structure which can maintain a list of currently executing background
  tasks – we suggest using a list. Also, now that you are maintaining a list of background tasks, add another built-
  in command ``jobs'' that outputs the name and PID of each task running in the background.
 \item Sometimes it's nice to have your program dump its output to a file rather than the console. In bash (and most
 shells) if you do \verb+cmd arg1 arg2 argN > filename+ then the output from cmd will go the file \cmd{filename}.
 If the file doesn't exist, it is created. If it does exist, it is truncated (removes all its data, allowing you to effectively
 overwrite it). Add this feature to your shell.
\end{itemize}

\section{Useful Information}
\subsection{Command line options} \label{cmdline}
Command line options are options that are passed to a program when it is initiated. For example, to get \cmd{ls} to
display the size of files as well as their names the \cmd{-l} option is passed to it. (e.g. \cmd{ls –l /home/me}).
In this example, \cmd{/home/me} is also a command line option which specifies the desired directory for which the contents
should be listed. From within a C program, command line options are handled using the parameters \cmd{argc} and
\cmd{argv} which are passed into main.
\begin{verbatim}
int main(int argc, char **argv)
\end{verbatim}
The parameter \cmd{argc} is a value that contains the number of command line arguments. This value is always at
least 1, as the name of the executable is always the first parameter. The parameter \cmd{argv} is an array of string
values that contain the command line options. The first option, as mentioned above, is the name of the
executed program. The program below simply prints each command line option passed in on its own line.
\begin{verbatim}
int main(int argc, char **argv) {
    int i;
    for(i=0; i<argc; i++)
        printf("Option %d is \"%s\"\n", i, argv[i]);
    return 0;
}
\end{verbatim}
If you are unfamiliar with how command line options work, enter the above program and try it with different
values. (e.g. \verb+./myprog I guess these are options+)
\subsection{Useful system and library calls}
The following system calls will be useful in this project. Read the corresponding man pages for those you are
unfamiliar with.

\begin{itemize}
 \item fork -- create a child process
 \item execvp -- replace the current process with that of the specified program
 \item waitpid -- wait for a child to exit (or get exit status)
 \item exit -- force the current process to exit, with the given return value
 \item chdir -- change working directory
 \item getcwd -- get current working directory
 \item getenv/setenv -- retrieve and set environment variables.
 \item perror -- display error messages based on the value of errno
 \item strcmp, strcpy, strcat -- string manipulation.
 \item open, dup2 -- for the extra credit
\end{itemize}

\section{Example and Test Cases}

This is an example test case and output for your progrem. Your output does not have to look exactly like this;
it is simply meant to give you some idea of how the shell should work and how output looks.
 A \verb=$= prompt indicates a bash prompt, used to execute your shell, and \verb+308sh>+ indicates your shell prompt.
Also, you may want to prefix all your output messages with some sort of identifier,
such as ``$>>>$”, to more easily determine which lines are yours and which came
from the executed program.

You should test all of the builtin commands in your shell, in addition to several commands that are not builtin.
You should test any error handling you use. You do not have to turn in any test code that you use, but don't expect
to find errors without testing!

Note that you do not have to output the process name on the ``exit'' line unless you are doing the extra credit; just the PID and
status is sufficient.
\\[10pt]
Shell command line:
\begin{verbatim}
$ ./shell -p "hello> "
hello> exit
$
\end{verbatim}
Simple execution:
\begin{verbatim}
308sh> /bin/ls
[977801] ls
shell.c  shell.o  shell
[977801] ls Exit 0
308sh> pwd
/home/danield/308/
308sh> cd
308sh> pwd
/home/danield/
308sh> cd 308
308sh> pwd
/home/danield/308/
308sh> ps
[977819] ps
   PID TTY          TIME CMD
  3590 pts/2    00:00:01 bash
977797 pts/2    00:00:00 shell
977819 pts/2    00:00:00 ps
[977819] ps Exit 0
308sh>
308sh> nonexistant
[977850] nonexistant
Cannot exec nonexistant: No such file or directory
[977850] nonexistant Exit 255
308sh> test 2 = 3
[978229] test
[978229] test Exit 1
308sh>
\end{verbatim}
Background processes:
\begin{verbatim}
308sh> sleep 1 &
[977999] sleep
308sh>
[977999] sleep Exit 0
308sh>

308sh> sleep 1 &
[977864] sleep
308sh> sleep 2
[977865] sleep
[977865] sleep Exit 0
[977864] sleep Exit 0
308sh>

308sh> sleep 10 &
[977943] sleep
308sh> kill 977943
[977945] kill
[977945] kill Exit 0
[977943] sleep Killed (15)
308sh>
\end{verbatim}


\end{document}
