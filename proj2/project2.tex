\documentclass[letterpaper,10pt]{article}
\usepackage{amsfonts}
\usepackage{amssymb}
\usepackage{amsmath}
\usepackage{amsthm}
\usepackage{enumitem}
\usepackage{mathrsfs}
\usepackage[utf8]{inputenc}

\addtolength{\voffset}{-1cm}
\addtolength{\hoffset}{-2cm}
\addtolength{\textwidth}{4cm}
\addtolength{\textheight}{2cm}
\newcommand{\cmd}[1]{\texttt{#1}}

%\usepackage[pdftex]{color}
%\usepackage[pdftex]{graphicx}
%\usepackage[pdftex]{hyperref}
%	\hypersetup{colorlinks}
%	\hypersetup{linkcolor=blue}

\title{CprE 308 Project 2: Multithreaded Server}
\author{Department of Electrical and Computer Engineering \\ Iowa State University}
\date{Spring 201j}

\begin{document}
\maketitle

This is a two-week programming project assignment, starting in the week of Mar 3-7, 2014,
and will due in the week of Mar 24-28, 2014.

\section{Submission}

Submit to Blackboard by due date (starting time for your lab session, March 25, 17, and 28,
respectively, 2014), include the following in a zip archive:

\begin{itemize}
 \item A cohesive summary of what you learned in the project. This should be no more than two paragraphs.
 \item Your source code; commented and formatted neatly. Use good programming practices
  (create methods to abstract large bodies of code and code common to multiple methods, only use global
  variables when you must, and check syscall or system dependant libcall return values for errors -- don't assume
  the OS will always succeed in fulfilling your requests.)
 \item A Makefile and instructions to compile the source (in a comment at the top of the .c file)
 \item The output of the provided test script when run on your program with a seed value of 0 for 10
 worker threads on 1000 accounts.
 \item A short report summarizing your findings from Part II.
 \item Include your summary, source code, Makefile, output and short report in a single archive on Blackboard.
 Name your archive appserver-lastname-section.tar.gz.
\end{itemize}

\paragraph{Late Policy}
You will lose 10\% per day for one week. After one week, no credit will be given.

\section{Project}
In this project, you will be creating a multithreaded server that manages access to a set of bank accounts.
The program is started with a certain number of accounts in the bank, and user requests can perform
transactions on these accounts, as well as query their current status. Note that the number of accounts
in the bank remains fixed through the duration of the program.

Each request requires access to multiple records from a database, and also requires some processing
time. Since many user requests may arrive simultaneously, servicing the requests sequentially, one after
another, would cause a high average latency for the users; servicing the requests in parallel is required
to keep response times small. Thus, your program has to use multiple threads to service the requests
simultaneously.

User requests are presented via the keyboard. When the user types in a request, the server records
the request and immediately presents the user with a transaction ID, and then continues to accept more
requests. At a later point of time, when the request has been processed, the results will be written out.
To avoid interfering with user input, the delayed responses will be written to a file rather than displayed
directly.

Tip: you can open a second terminal and use \verb+tail -f file+ to follow the delayed output.


\section{Requirements}
The server should be written using C on a Linux machine, using the pthreads package. During initialization,
the server should create a specified number of worker threads, which service user requests. The number
of worker threads will be specified by the user as argument at the command line.

\subsection{Threads Within the Program}
The program should contain two types of threads, a ``main" thread, and one or more ``worker" threads. The
main thread will initialize any state of the server and create all other ``worker" threads.
Each worker thread
will service one request at a time. After servicing a request, the worker thread will service another request,
and so on, until the end of the server. The program can be launched (and usually will be launched) with
more than one worker thread. Thus, there are typically many worker threads, simultaneously processing
different requests.

The syntax used to launch the program will be

\begin{verbatim}
$ appserver <# of worker threads> <# of accounts> <output file>;
\end{verbatim}

For example, to run a server with 4 worker threads and 1000 accounts that outputs to the file ``responses", you would run

\begin{verbatim}
$ appserver 4 1000 responses
\end{verbatim}

Assume that the balance of each account is initialized to zero. Each account is assigned an ``account id".
If there are $n$ accounts in total, the account ids take consecutive values in the range $1$ till $n$.


\subsection{Requests}
A request will be a single line of input. The first word of this line
will identify the type of request. You
are required to handle 3 types of requests: transactions, balance checks, and program exit.

For transactions and balance checks, there should be an immediate response to the user, of the form:
\begin{verbatim}
ID <requestID>
\end{verbatim}
The request ID should start at 1 and should be incremented each time a new request is issued.

\subsubsection{Balance Check}
The request line will be of the format:
\begin{verbatim}
CHECK <accountid>
\end{verbatim}
An account ID is a positive (nonzero) integer less than the specified
maximum account ID. The output of this request written to the result file have the following form:
\begin{verbatim}
<requestID> BAL <balance>
\end{verbatim}

\subsubsection{Transaction}
\begin{verbatim}
TRANS <acct1> <amount1> <acct2> <amount2> <acct3> <amount3> ...
\end{verbatim}
There may be from 1 to 10 accounts in any single transaction.
The amounts will be signed integers representing the transaction amount in cents.
For each pair consisting of an account id followed by an amount, the amount
should be added to the account balance.

Significantly, all transfers within a transaction should happen as a single unit; This means each
transaction is processed in a single thread, and no other thread may access the accounts involved in this
transaction while the transaction is being processed. \emph{If any account does not have a sufficient balance to
satisfy the transaction, the entire transaction is voided and all accounts should return to their state at the
start of the transaction.}

If a transaction was successful, your program should write a result of the following form to the output file:
\begin{verbatim}
<requestID> OK
\end{verbatim}

If the transaction was unsuccessful because some account did not have sufficient funds, the following
should be written to the output file:
\begin{verbatim}
<requestID> ISF <acctid>
\end{verbatim}
where acctid is the account that had insufficient funds (there may be several; you only need to identify
one).


\subsubsection{End}
\begin{verbatim}
END
\end{verbatim}
No further commands will be processed from the user. All currently queued commands should
be processed, and then the program should exit. You do not need to print a request ID
for this command, but can if it is more convenient to do so.

Errors such as invalid input can be handled when a request is entered or in the worker thread when
it is processed; they should not result in a successful transaction or a crash. You may also choose how to
inform the user of the error.


\subsection{Account Database Interface}

We will supply you with an interface to an account database. This database stores all the accounts
and manages access to them. You can interact with the accounts in the database through three simple
functions.
\begin{verbatim}
void initialize_accounts( int n );
\end{verbatim}

This function should be called once at the start of your program. It takes as a parameter an int n - the
number of bank accounts. The function initializes the database with n accounts with IDs from 1 to n,
and sets the value of each account to 0.
\begin{verbatim}
int read_account( int id );
\end{verbatim}

This function returns the value of account id.
\begin{verbatim}
void write_account( int id, int value);
\end{verbatim}

This function writes value to the supplied account.

You will be given the Bank.h and Bank.c files which contain these functions. You should not modify
the contents of these files. Note that these functions provide no safety against concurrent read/writes or
protection against writing to invalid accounts; that is the responsibility of your program to manage.

Because the accounts are all in-memory, while they would be stored on disk in a real banking system,
these functions also artificially in
ate the access delay so that they are similar to what is experienced in
a real system.


%\subsection{XXX Processing Delays}
%Because the accounts are all in-memory, the delay from many actions that would normally
%take time in a real system must be emulated. Also, since we want to try to detect deadlocks,
%locking will have additional delays associated with it to try to more easily produce a deadlock
%that might happen very rarely with no delays.
%Execute the sleep after performing the action. \\
%\begin{tabular}{ll}
% \hline
% Action & Time (in $\mu$s) \\
% \hline
% Lock Mutex    &  100 \\
% Unlock Mutex  &  100 \\
% Read balance  & 5000 \\
% Write balance & 7000 \\
% \hline
%\end{tabular}

\subsection{Timing}
In order to evaluate how long requests take to process, you are \textbf{required} to include
the system time when a request is received by the main thread, in addition to the
time that the request has finished. To get more than second-level accuracy, use
the \verb+gettimeofday()+ function with \verb+NULL+ as the second argument; this
will provide microsecond timestamps. Include the timestamps in the same line as the
request output, preceded by the word ``TIME''. Use the printf format \verb+"%d.%06d"+
to print out seconds and microseconds without needing to use floating-point numbers.

\subsection{Command Buffer}
%Because the number of worker threads is limited, your main program cannot wait for a worker thread
%to begin executing a request before returning to process user input. To handle this, you should create a buffer of
%requests that will be processed in the worker threads. The worker threads should wait
%until there is a new user request to be served. When a new user request arrives, one of the worker
%threads should wake up and service the request. After servicing one request, the worker again checks for new
%requests in the buffer, and if none are currently present, it goes back to waiting for more
%requests.
%
%Because the buffer will be accessed by multiple threads, it needs to be protected by a mutex.
%The buffer can be implemented by using a linked list or a circular array. If you use an array,
%you should make it much larger than the number of worker threads (1024 entries), and you
%need to handle the buffer wrapping around (due to more than 1024 total requests), or becoming
%full (which will happen less often). If using a linked list, remember to free any memory you
%allocate. It is easier to use a linked list for this type of buffer.
%
%Only one worker thread should wake up to serve a request; it is a waste of resources to wake
%up all threads in this case. However, there is one exception: all threads must be woken so that they
%can terminate when the END command is sent.

You should have a ''Linked List" data structure to hold those requests that have been received but not
processed yet. When a request is received by the main thread, the thread should add it to the linked list.
Each worker thread should pickup requests from the linked list and process them. If there are no requests
in the linked list, the worker thread must wait until there is one. Remember to free any memory you
allocate.

Because the buffer will be accessed by multiple threads, it needs to be protected by a mutex.



\section{Tips}

\subsection{Account Storage}
%A structure such as the following is all that is required for an account:
%\begin{verbatim}
%struct account {
%    pthread_mutex_t lock;
%    int value;
%};
%\end{verbatim}
%You can assume that the number of accounts is small enough that they will all fit in memory.

Since the accounts themselves maybe accessed by multiple threads, we should ensure that concurrent
access to the same account is prevented. One way to protect the accounts from simultaneous access is
through using a mutex for each account.

For example, an account maybe represented by the following structure:
\begin{verbatim}
struct account {
    pthread_mutex_t lock;
    int value;
};
\end{verbatim}


\subsection{Deadlocks}
%To correctly commit a transaction, a thread must request multiple locks at once.
%If this program is not written with deadlocks in mind, it will be fairly easy to trigger one.
%It is possible to avoid any possibility of deadlocks by ensuring that one of the preconditions
%for a deadlock does not hold; you should ensure this in your program.

To correctly execute a transaction, a thread must hold multiple locks at the same time. For example, if a
transaction has three accounts in it, the worker thread executing that transaction must hold locks for all
three accounts at the same time. This behavior can easily lead to deadlocks unless our program is written
carefully.

Your program should not lead to a deadlock.
Refer to the lecture notes on deadlocks and the notes on pthread mutexes for possible solutions.


\subsection{File Output}
%The C standard library functions for \verb+FILE*+ objects are threadsafe; the FILE structure
%contains a mutex that all output functions acquire before using the file. However, if you
%execute two successive \verb+fprintf()+ calls, there is no guarantee that they will
%output in order. The functions \verb+flockfile(FILE*)+ and \verb+funlockfile(FILE*)+ are
%available if you need to ensure correctly ordered output.
%
%File locking is not needed for this program if each line is written in a single call
%to fprintf; however, if you create utility functions to output some parts, it is needed.

You can use the \verb+fprintf()+ function to output to a file; its usage is similar to that of \verb+printf+.

The C standard library functions for FILE* objects are threadsafe; the FILE structure contains a
mutex that all output functions acquire before using the file. However, if you execute two successive
\verb+fprintf()+ calls, there is no guarantee that they will output in order. The functions \verb+flockfile(FILE*)+ 
and \verb+funlockfile(FILE*)+ are available if you need to ensure correctly ordered output.

\subsection{Test Script}

A Perl test script testscript.pl is included for testing. To run the script, run
\begin{verbatim}
./testscript <program> [<nthreads> [<naccounts> [<seed>]]]
\end{verbatim}

For example, to test your program with 10 worker threads, 1000 accounts, and a seed of zero,use:
\begin{verbatim}
./testscript ./appserver 10 1000 0
\end{verbatim}

When the test script is downloaded, it will not have execute permissions. To give execute permissions
to the file, run the following command
\begin{verbatim}
chmod u+x ./testscript
\end{verbatim}

\section{Example}
Input/output at the console. Lines beginning with a \verb+>+ are input, lines beginning with a \verb+<+ are output.

\begin{verbatim}
$ appserver 4 1000 requests
> CHECK 1
< ID 1
> TRANS 1 100000 2 100000
< ID 2
> TRANS 1 -10000 5 10000
< ID 3
> TRANS 2 -20000 4 10000 7 10000
< ID 4
> CHECK 1
< ID 5
> TRANS 2 -2000 1 1000
< ID 6
> TRANS 1 -10000 4 -10100 5 20100
< ID 7
> CHECK 4
< ID 8
\end{verbatim}
The file ``requests'' might then contain:
\begin{verbatim}
1 BAL 0 TIME 1224783296.348723 1224783296.913123
2 OK TIME 1224783297.348254 1224783298.034256
3 OK TIME 1224783297.349756 1224783298.068902
4 OK TIME 1224783298.350484 1224783298.647822
5 BAL 90000 TIME 1224783298.348467 1224783298.609814
7 ISF 4 TIME 1224783299.548467 1224783299.741932
6 OK TIME 1224783299.478867 1224783299.834902
8 BAL 10000 TIME 1224783302.899871 1224783303.002389
\end{verbatim}
Since the transactions are being processed in multiple threads, the output may not be
in the same order as the input; this is the reason request IDs are used.

You do not need to ensure that the results are the same as if the commands were executed in a
single thread, but you do need to ensure that the results are correct for \emph{some}
ordering of transactions. For example, summing the amounts of all OK transactions should
be equal to the balance in a given account.

\section{Part II - Locking Granularity Exploration}

In this section you will explore a trade-off between coarse and fine grained locking. Until now, you have
used fine-grained mutexes to protect the accounts by having a separate mutex for each account. This
allows the most control over the individual accounts, but it can come at a performance penalty to process
each individual mutex lock and unlock. In this section you will modify the project to instead use a single
mutex to protect the entire bank to emulate coarse-grained locking.

\subsection{Coarse-Grained Locking}

Create a copy of your project. Name the new project appserver-coarse.c Modify the new version to lock
the entire bank (every account) for each request. Note that this uses a significantly smaller number of
mutex lock and unlock operations, but also leads to a significantly smaller concurrency among the worker
threads.

\subsection{Performance Measurement}

Use the time command to measure the running time of both programs. For consistency, use the provided
test script with the same parameters as before. The following commands can be used to determine
performance:
\begin{verbatim}
time ./testscript.pl ./appserver 10 1000 0
time ./testscript.pl ./appserver-coarse 10 1000 0
\end{verbatim}

Run the test multiple times on each program to produce an average. Be sure to use the "real" time
from the output results.

\subsection{Summary}

Summarize your findings in a short report. Include your results in a table and be sure to address the
following points:
\begin{itemize}
\item Which technique was faster - coarse or fine grained locking?
\item Why was this technique faster?
\item Are there any instances where the other technique would be faster?
\item What would happen to the performance if a lock was used for every 10 accounts? Why?
\item What is the optimal locking granularity (fine, coarse, or medium)?
\end{itemize}


\end{document}
