\documentclass[letterpaper,10pt]{article}
\usepackage{amsfonts}
\usepackage{amssymb}
\usepackage{amsmath}
\usepackage{amsthm}
\usepackage{enumitem}
\usepackage{mathrsfs}
\usepackage[utf8]{inputenc}

\addtolength{\voffset}{-1cm}
\addtolength{\hoffset}{-2cm}
\addtolength{\textwidth}{4cm}
\addtolength{\textheight}{2cm}
\newcommand{\cmd}[1]{\texttt{#1}}

%\usepackage[pdftex]{color}
%\usepackage[pdftex]{graphicx}
%\usepackage[pdftex]{hyperref}
%	\hypersetup{colorlinks}
%	\hypersetup{linkcolor=blue}

\title{CprE 308 Laboratory 6: Memory Management}
\author{Department of Electrical and Computer Engineering \\ Iowa State University}
\date{Spring 2014}

\begin{document}
\maketitle
\section{Submission}
Include the following in your lab report:
\begin{itemize}
 \item Your full name and lab section in the upper right corner of the first page in large print.
 \item A summary of what you learned in the lab session. This should be no more than two
paragraphs. Try to get at the main idea of the exercises.
 \item A table that shows the page faults for each of the algorithms using the three sequences.
 \item Submit your summary, write-ups, and *changed* source code as a written report.
\end{itemize}


\section{Introduction and background}
In this assignment, you will write a program that simulates three virtual memory page
replacement algorithms: Optimal (OPT), Least Recently Used (LRU) and First-In-First-Out (FIFO).
Here is a brief description of the three algorithms that you are going to implement in your code.
More details can be found in the text book.

\paragraph{{\bf OPT}:} when a page needs arrives into the memory, the OS replaces the page whose next use will
occur farthest in the future. Note that this algorithm cannot be implemented in the general-purpose OS,
because the pattern of future page references is usually unknown.

\paragraph{{\bf FIFO}:} This is the simplest page-replacement algorithm. When a page needs to be replaced, the
page that is chosen is the one which arrived earliest into memory, among all the page frames
currently in memory. FIFO performs poorly in practice since it ignores the usage history of a
page and only focuses on the time of arrival. Thus, it is rarely used in its unmodified form.

\paragraph{{\bf LRU}:} The page to be swapped out is the page whose most recent access time is the earliest. LRU
works on the idea that pages that have been used in the past few instructions are most likely to
be used in the next few instructions too.

\section{Lab Description}
We assume an application that uses a total of 128 pages, some of which are on disk, and some
in memory. The memory has 16 page frames, each of which can hold a single page. Initially, the
memory is empty, and any page access will result in a {\bf page fault}, which means that the page is
not in the memory and must be collected from the disk. After that, for each page access, the OS
first checks the memory. If the page is in memory, then it is a {\bf page hit} where the OS access the
page from the memory. If the page is not in the memory, then it results in a page fault, where
the page is retrieved from the disk. On a page fault, if the memory is not full (there is an empty
page frame), then the page is placed in an empty page frame. If the memory is full (i.e. all 16
page frames are occupied), then the OS should find a page to be replaced. The replacement
algorithms that we consider are OPT, LRU, and FIFO: your task is to implement these three
algorithms.

We maintain a structure for each page frame, which contains the (1)ID of the page in this page
frame (2) most recent access time of the page, and (3)arrival time of the page. On a page hit, the
OS updates the time of most recent access. On a page fault, the OS brings the new page to the
memory and records the last access and memory arrival times to the same value which is the
current time. The time is the index of the page in the input file. Here is the structure
representing a page frame.
\begin{verbatim}
typedef struct {
    int page_id;
    int time_of_access;
    int time_of_arrival;
} PageFrame;
\end{verbatim}

\subsection{Inputs}
The inputs to the program are page accesses sequences; each sequence consists of 10000
memory pages accesses. The sequence contains the IDs of the pages that are accessed by the
application. The time of the access is equal to the index of the entry.
\begin{enumerate}
\item First sequence will be sequential (SEQ). The application accesses the pages 1,.. ,128 in
order, and this sequence is then repeated.
\item The second sequence is random (RAN): there will be 10000 page IDs, each of them a
random number between 1 and 128.
\item The third sequence (LR) is constructed to obey a “locality of reference pattern”. With
probability 0.9 the next page to be accessed is one of the 5 pages that were recently
been accessed, and with probability 0.1 a page among the 128 pages will be accessed.
\end{enumerate}

\subsection{Task}
Study the code and understand the different parts involved. Then implement the two functions
``PRAlgo\_OPT" for the optimal algorithm, and ``PRAlgo\_LRU", for the ``Least Recently Used"
algorithm. One replacement algorithm, ``First in First Out" is already implemented in the
function ``PRAlgo\_FIFO". You also need to implement your own custom algorithm –
``PRAlgo\_CUST". This algorithm should be designed by you and the goal is to beat the LRU
algorithm on the Locality of reference access sequence, you may use whatever information is
available to the PRAlgo\_CUST function to achieve this.

This code contains three parts. The first part is the generation of the different access sequences,
which we have written. The second part is the code which reads these sequences of memory
pages and checks whether each page is in the memory or not. Finally, a function for deciding
which page in memory should be replaced; this function is called only in case there is a page
fault AND there is no space for additional pages in memory. You should write the code for the
third part.

\subsection{Output}
For each of the given three sequences, should show the number of page faults using each of the
four algorithms. Therefore, we expect 12 numbers, three per algorithm.

\subsection{Grading}
The grader will check your code on three sequences and see if your code gives the correct 9
results for the FIFO, LRU, and OPT algorithms, and whether your CUST algorithm beats LRU on
the Locality of Reference access sequence.

The grading criteria is as follows:

\begin{itemize}
\item Summary: 10pts
\item Compile: 5pts
\item Code commented: 5pts
\item Replacement algorithms' implementation:
\begin{verbatim}
   OPT: 18pts
   LRU: 18pts
   CUST: 18pts
\end{verbatim}
\end{itemize}

Testing on three sequences:
\begin{itemize}
\item Correct table generation for OPT, LRU, and FIFO: 18pts
\item Beats LRU on the Locality of Reference access sequence: 8pts
\end{itemize}

The total points are 100pts

\end{document} 
